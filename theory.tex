\documentclass[12pt, letterpaper]{article}
% Packages
\usepackage{amsmath}
\usepackage{amsfonts}
\usepackage{amsthm}
\usepackage{hyperref}
\newtheorem{lemma}{Lemma}

\setlength{\parindent}{0em}
\setlength{\parskip}{0em}
\title{An Integer Programming Solver for The Wicker Rod Optimization Problem}
% Document
\begin{document}
\maketitle

\section{Requirements}
To manufacture a rattan box, 4 pieces of length 35, 4 pieces of length 30 and 4 pieces of length 20 are needed. For that, rods of length 330 are provided. The goal of the producer is to find a batch size and the "recipe" that will allow the production of that batch while minimizing the absolute wicker waste.



\section{Formalization of the problem}
We start with a little bit of notation:
\begin{itemize}
    \item We will call $l_1, l_2, l_3$ to the three lengths of the rods, sorted descending.
    \item Let $l$ be the total length of the provided rods.
    \item Each "recipe" will amount giving to describing the ways of cutting the provided rods of length $l$; formally $p = (l^1\dots,l^k)$ where $l^1 + \dots + l^k = l$.
\end{itemize}

In order to formalize the problem, we define the notion of \textit{optimal partition}
as a partition $(l^1,\dots,l^r)$ where $l^1 \geq \dots \geq l^{r-1} > l^r$ y $l^1,\dots,l^{r-1}\in\{l_1,l_2,l_3\}$. The sorting condition is non-essential and it only simplifies the later development (and the clarify of presentation for the manufacturer), the important thing is that all pieces corresponds to lengths in  ${l_1,l_2,l_3}$ and that there will be only one clipping piece, which won't be possible to use.

\vspace{10px}


Let $P$ be the set of all partitions, and let $\hat{P}$ the set of just the optimal ones. It will become clear that it's possible to compute easily the set $\hat{P}$; so we will be able to write $\hat{P} = (p^1,\dots,p^k)$, where $k$ is the total quantity of optimal partitions for the parameters $(l, l_1, l_2, l_3)$.

\vspace{10px}

For each partition $p$, we consider for integer quantities:
\begin{enumerate}
    \item $N_1(p), N_2(p)$ y $N_3(p)$ that amount to the multiplicities of $l_1$, $l_2$ and $l_3$ respectively in the partition $p$.
    \item $C(p) = l^r$ si $p = (l^1,\dots,l^r)$. That is, $C(p)$ is the clipping length of partition $p$.
\end{enumerate}
\vspace{10px}
We will finally introduce multiplicities $m_1, m_2, m_3$ that correspond to the quantities needed for manufacturing a production unit. In the requirement for the rattan box, we trivially have $m_1 = m_2 = m_3$, but this seems just like an accidental fact and also there are no great difficulties in taking $m_1, m_2, m_3 \in\mathbb{N}$ as wished.
\vspace{10px}
\par
All that said, we state the optimization problem:
\par
\vspace{10px}
Find $(n_1,..,n_k)\in\mathbb{N}^k$ that minimizes $\sum_{i=1}^n C(p_i)\times n_i$, sujebject to the constrains
\begin{enumerate}
    \item $m_{j_1}(\sum_{i=1}^k N_{j_0}(p^i)\times n_i) = m_{j_0}(\sum_{i=1}^k N_{j_1}(p^i))\times n_i)\quad\forall j_0, j_1\in \{1, 2, 3\}$
    \item $n_1 + \dots + n_k < n_{bound}$
\end{enumerate}
being $n_{bound}$ the maximum admissible amount for the batch. \\


\vspace{10px}
\par

Notice that the vector $(n_1, \dots, n_k)$ returned by the algorithm may not be directly translated into a suitable for a given batch size. For the batch size $b$ the vector must accomplish $bm_{j} = (\sum_{i=1}^k N_{j}(p^i)\times n_i)\;\forall j\in\{1,2,3\}$.
\vspace{10px}\\
For that purpose, we start by defining
$$m_1' = \text{lcm}(m_1, N_1)$$
- where $N_j$ is a notation for $\sum_{i=1}^k N_{j}(p^i)\times n_i$- , as the quantity we need to manufacture for length $l_1$ and then we can set the batch size to $\frac{m_1'}{m_1}$. We have:
\begin{itemize}
    \item $bm_1 = \frac{m_1'}{m_1}m_1 = m_1' = \frac{m_1}{\text{gcd}(m_1, N_1)}N_1$ by property of lcm.
    \item $bm_2 = \frac{m_1'}{m_1}m_2 = \frac{N_1}{\text{gcd}(m_1, N_1)}m_2 = \frac{m_2N_1}{\text{gcd}(m_1, N_1)} = \frac{m_1N_2}{\text{gcd}(m_1, N_1)} = \frac{m_1}{\text{gcd}(m_1, N_1)}N_2$
    \item $bm_3= \frac{m_1'}{m_1}m_3 = \frac{N_1}{\text{gcd}(m_1, N_1)}m_3 = \frac{m_3N_1}{\text{gcd}(m_1, N_1)} = \frac{m_1N_3}{\text{gcd}(m_1, N_1)} = \frac{m_1}{\text{gcd}(m_1, N_1)}N_3$
\end{itemize}
\vspace{10px}
\par
From this we can see that the vector $(\frac{m_1}{\text{gcd}(m_1, N_1)}n_1, \dots, \frac{m_1}{\text{gcd}(m_1, N_1)}n_k)$ accomplishes the manufacturing objective with a batch size of size $\frac{m_1'}{m_1}$.


\section{Tackling the problem using IP}

What we see next is that, if fixing $n < n_{bound}$, we can find an equivalent IP problem for the original problem.
\par
\vspace{10px}
We start by defining the matrix $A\in\mathbb{N}^{2\times k}$ as:
\begin{equation}
    \begin{pmatrix}
        m_2N_1(p_1)  - m_1N_2(p_1)& \cdots & m_2N_1(p_k) - m_1N_2(p_k)\\
        m_1N_2(p_1)  - m_2N_1(p_1)& \cdots & m_1N_2(p_k) - m_2N_1(p_k)\\
        m_3N_2(p_1)  - m_2N_3(p_1)& \cdots & m_3N_2(p_k) - m_2N_3(p_k)\\
        m_2N_3(p_1)  - m_3N_2(p_1)& \cdots & m_2N_3(p_k) - m_3N_2(p_k)\\
        1 & \cdots & 1
    \end{pmatrix}
\end{equation}
y el vector $\vec{c}\in\mathbb{N}^k$
\begin{equation}
    \begin{pmatrix}
        C(p_1) \\
        \vdots \\
        C(p_k) \\
    \end{pmatrix}
\end{equation}
\par
\vspace{10px}
We can state the following problem as a classic IP problem:
\par
\vspace{5px}
Find $\vec{n}\in\mathbb{Z}^k$ that minimizes $\vec{c}\cdot\vec{n}$, subject ot the constrains:
\begin{enumerate}
    \item $A\vec{n} \leq (0, 0, n_{bound})^t$
    \item $\vec{n}\geq\vec{0}$
\end{enumerate}

\vspace{10px}\par
It's clear that:
\begin{enumerate}
    \item Restriction 2. makes $\vec{n}\in\mathbb{N}^k$
    \item It is clear that
    \par
    \begin{center}
        \begin{tabular}{c c c}
        $(A\vec{n})_0\leq 0\wedge (A\vec{n})_1 \leq 0$  & $\iff$ &  $\sum_{i=1}^k (m_2N_1(p_i)  - m_1N_2(p_i))\times n_i = 0$ \\
         & $\iff$ & $m_2(\sum_{i=1}^k N_1(p_i)\times n_i) = m_1(\sum_{i=1}^k N_2(p_i)\times n_i)$ \\ 
    \end{tabular}
    \end{center}
    Similarly, we get
    \begin{center}
        \begin{tabular}{c c c}
            $(A\vec{n})_2\leq 0\wedge (A\vec{n})_3\leq 0 \iff m_2(\sum_{i=1}^k N_1(p_i)\times n_i) = m_1(\sum_{i=1}^k N_2(p_i)\times n_i)$.
        \end{tabular}
    \end{center}
    Finally, we get the restriction 1. of the original problem on $j_0 = 1, j_1 = 3$ for free:
    \begin{itemize}
        \item By multiplying the first equality by $m_3$ we get $$m_3m_2(\sum_{i=1}^k N_1(p_i)\times n_i) = m_3m_1(\sum_{i=1}^k N_2(p_i)\times n_i)$$
        \item By multiplying the second equality by $m_1$ we get $$m_1m_3(\sum_{i=1}^k N_2(p_i)\times n_i) = m_1m_2(\sum_{i=1}^k N_3(p_i)\times n_i)$$
        \item By transitivity and cancelling out $m_2$, we get
        $$m_3(\sum_{i=1}^k N_1(p_i)\times n_i) = m_1(\sum_{i=1}^k N_3(p_i)\times n_i)$$
        which implies restriction 1. of the original problem.
    \end{itemize}
    This completes the equivalence for condition 1.
    
    \item It's clear that $(A\vec{n})_2 \leq n_{bound}$ means exactly $n_1 + \dots + n_k < n_{bound}$, which is equivalent to restriction 1. of the original problem.
    
\end{enumerate}

We can conclude that the problem we just presented is therefore equivalent to the original problem that we want to solve.

\section{Implementation}

The implementation of the IP problem stated above is done with the \href{https://www.ibm.com/analytics/cplex-optimizer}{CPLEX} software. 

The only caveats that we can mention from the implementation process are:
\begin{itemize}
    \item Generating the optimal partitions is done following a left-to-right "greedy" algorithm, which will return them in the standard lexicographic order.
    \item As the trivial solution ($\vec{n} = 0$) is always an optimal solution (and also the one chosen by the solver), we needed to add a final row $(-1 \cdots -1)$ with $-1$ as the corresponding restriction; which will enforce the solution not being trivial.
\end{itemize}

The implementation can be accessed on \href{https://github.com/damif94/wicker_rod_problem}{Github}.

\section{Further research}
The fact that we are using the set $\hat{P}$ instead of the plain $P$ is a relaxation for the problem we actually stated informally in section 1, since any solution using a non-optimal partition $p$ can be done at least as effectively by replacing it with $\hat{p}$ that has at least the same multiplicities at less cost (sorting the lengths in $p$ and then filling the original clipping respecting the optimality).
\par
\vspace{10px}
But, at first sight, the equality of condition $1.$ in the original formulation from section 2 is not a relaxation, since we could, in theory, have a solution that wastes less wicker by affording some pieces of lengths in $\{l_1, l_2, l_3\}$ to be wasted, along with the clippings.
\par
\vspace{10px}
Regarding this last point, it's probably not difficult to found a reasonable bound to the inequalities that serve as valid relaxations, and rewrite the IP problem accordingly using this bound for the values $c_0, \dots, c_3$ instead of just 0.
\end{document}%
